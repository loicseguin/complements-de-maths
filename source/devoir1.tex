\documentclass[]{article}
\usepackage[T1]{fontenc}
\usepackage{lmodern}
\usepackage{amssymb,amsmath}
\usepackage{wasysym}
\usepackage{ifxetex,ifluatex}
\usepackage{fixltx2e} % provides \textsubscript
% use microtype if available
\IfFileExists{microtype.sty}{\usepackage{microtype}}{}
\ifnum 0\ifxetex 1\fi\ifluatex 1\fi=0 % if pdftex
  \usepackage[utf8]{inputenc}
\else % if luatex or xelatex
  \usepackage{fontspec}
  \ifxetex
    \usepackage{xltxtra,xunicode}
  \fi
  \defaultfontfeatures{Mapping=tex-text,Scale=MatchLowercase}
  \newcommand{\euro}{€}
\fi
% Redefine labelwidth for lists; otherwise, the enumerate package will cause
% markers to extend beyond the left margin.
\makeatletter\AtBeginDocument{%
  \renewcommand{\@listi}
    {\setlength{\labelwidth}{4em}}
}\makeatother
\usepackage{enumerate}
\ifxetex
  \usepackage[setpagesize=false, % page size defined by xetex
              unicode=false, % unicode breaks when used with xetex
              xetex]{hyperref}
\else
  \usepackage[unicode=true]{hyperref}
\fi
\hypersetup{breaklinks=true,
            bookmarks=true,
            pdfauthor={},
            pdftitle={Devoir 1},
            colorlinks=true,
            urlcolor=blue,
            linkcolor=magenta,
            pdfborder={0 0 0}}
\setlength{\parindent}{0pt}
\setlength{\parskip}{6pt plus 2pt minus 1pt}
\setlength{\emergencystretch}{3em}  % prevent overfull lines
\setcounter{secnumdepth}{0}

\title{Devoir 1}
\author{}
\date{}

\begin{document}
\maketitle

\textbf{Date de remise : le jeudi 30 septembre 2012}

\begin{enumerate}
\item
  Voici deux ``preuves'' qui contiennent évidemment des erreurs.

  \begin{description}
  \item[\textbf{3 = 0}]
  On commence à partir de l'expression

  \[x^2 + x + 1 = 0.\]

  En soustrayant $1$ de chaque côté de l'égalité, on obtient

  \[x^2 + x = -1.\]

  En multipliant l'équation :eq:`eq1` par $x$ de part et d'autre de
  l'égalité, on obtient

  \[x^3 + x^2 + x = 0\]

  et en remplaçant l'équation :eq:`eq2` dans l'équation :eq:`eq3`, on
  trouve

  \[x^3 - 1 = 0.\]

  De cette dernière équation on tire que $x = 1$. Si on remplace cette
  valeur de $x$ dans l'équation :eq:`eq1`, on obtient $3
  = 0$. CQFD.
  \item[\textbf{1\$ = 1\textcent}]
  \begin{eqnarray*}
  1 \$ &=& 100 \cent \\
  &=& (10 \cent)^2 \\
  &=& (0,1 \$)^2 \\
  &=& 0,01 \$ \\
  &=& 1 \cent
  \end{eqnarray*}

  CQFD
  \end{description}

  Pour chacune de ces preuves, déterminer l'erreur et l'expliquer.
\item
  Swokowski, p. 13 \#2
\item
  Swokowski, p. 13 \#10
\item
  Swokowski, p. 13 \#16
\item
  Swokowski, p. 13 \#34
\item
  \begin{enumerate}[a)]
  \item
    Écrire $2.3\overline{416}$ sous forme fractionnaire.
  \item
    Écrire $11/14$ sous forme décimale.
  \end{enumerate}
\end{enumerate}

\end{document}
